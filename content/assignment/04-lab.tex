\documentclass[11pt]{article}




\usepackage[sfdefault]{FiraSans} %% option 'sfdefault' activates Fira Sans as the default text font
\usepackage[T1]{fontenc}
\renewcommand*\oldstylenums[1]{{\firaoldstyle #1}}

\usepackage{natbib}
\usepackage[french,english]{babel}
\usepackage{numprint}
\usepackage{multirow}
\usepackage{rotating}
\usepackage{fancyhdr}
\usepackage{booktabs}
\usepackage{multicol}
\usepackage{hyperref}\hypersetup{colorlinks=true}

\usepackage{amsmath,amssymb,amsfonts,textcomp}
\usepackage{color}
\usepackage{calc}
 \setlength{\tabcolsep}{8pt}
\usepackage{setspace}
\onehalfspacing
\usepackage{longtable}
\usepackage{graphicx}
\usepackage[margin=1in]{geometry}
\setlength{\parindent}{0pt}
\usepackage[bottom]{footmisc}
\pagestyle{fancy}
\usepackage{titlesec}
\usepackage{lipsum}
\usepackage{cancel}
\usepackage{multicol}

\usepackage{amsmath,amssymb}
\usepackage{lmodern}
\usepackage{iftex}
\ifPDFTeX
  \usepackage[T1]{fontenc}
  \usepackage[utf8]{inputenc}
  \usepackage{textcomp} % provide euro and other symbols
\else % if luatex or xetex
  \usepackage{unicode-math}
  \defaultfontfeatures{Scale=MatchLowercase}
  \defaultfontfeatures[\rmfamily]{Ligatures=TeX,Scale=1}
\fi

\titleformat{\section}
  {\normalfont\Large\scshape\bfseries}{\thesection}{1em}{}
  \titlespacing{\section}{0pt}{10pt}{0pt}
\titleformat{\subsection}
  {\normalfont\bfseries}{\thesection}{1em}{}
  \titlespacing{\subsection}{0pt}{6pt}{0pt}
\providecommand{\tightlist}{%
  \setlength{\itemsep}{0pt}\setlength{\parskip}{0pt}}\newenvironment{itemize*}%

  
\lhead{EC200 - Econometrics and Applications}
\rhead{Version: \today}
\setlength\parskip{0.10in}
\begin{document}
\thispagestyle{plain}
\singlespacing


Version: \today \hfill Fall 2021\\
EC200: Econometrics and Applications
%\vspace{1.5cm}
\begin{center}
\Large{\textbf{Lab 4: Loops and Multivariate Regression}}\\
\end{center}
\bigskip


%{
%\setcounter{tocdepth}{3}
%\tableofcontents
%}

\hypertarget{materials}{%
\subsection*{Materials}\label{materials}}
\addcontentsline{toc}{subsection}{Materials}

\begin{itemize}
\tightlist
\item
  \href{https://ec200f21.netlify.app/assignment/materials/cps_2016.dta}{\texttt{cps\_2016.dta}}
\item
  Do-file template
  \href{https://ec200f21.netlify.app/assignment/materials/labtemplate_f21.do}{\texttt{labtemplate.do}}
\item
  Looping exercise
  \href{https://ec200f21.netlify.app/assignment/materials/loop_example.do}{\texttt{loop\_example.do}}
\end{itemize}

\hypertarget{objectives}{%
\subsection*{Objectives}\label{objectives}}
\addcontentsline{toc}{subsection}{Objectives}

Today we're going to work with some new data, \texttt{cps\_2016.dta},
which contains information from the
\href{https://cps.ipums.org/cps/}{2016 Current Population Survey}.

By the end of this lab, you should be able to complete the following
tasks in Stata:

\begin{itemize}
\item
  Estimate and interpret multiple linear regression in levels, using
  continuous and binary independent variables, and use
  heteroskedasticity-robust standard errors.
\item
  Interpret the results of multivariate linear regressions in terms of
  statistical and economic significance
\item
  Practice generating binary variables from categorical measures
\item
  Set up basic loops
\item
  Use \texttt{xi} and \texttt{i.} prefix to include a lot of binary
  indicator variables at once.
\end{itemize}

\hypertarget{key-commands}{%
\subsection*{Key commands}\label{key-commands}}
\addcontentsline{toc}{subsection}{Key commands}

\begin{longtable}[]{@{}
  >{\raggedright\arraybackslash}p{(\columnwidth - 2\tabcolsep) * \real{0.56}}
  >{\raggedleft\arraybackslash}p{(\columnwidth - 2\tabcolsep) * \real{0.44}}@{}}
\toprule
command & description \\
\midrule
\endhead
\texttt{regress\ var1\ var2...} & Estimate a regression, with
\texttt{var1} as the dependent variable and all others as the
independent \\
variable(s) & \\
\texttt{tabulate\ var1,nolabel} & Tabulate variables \emph{without}
labels \\
\texttt{replace\ var1\ =\ .\ if\ var1\ ==\ 999999} & Replace
\texttt{var1} as missing (using a dot) if \texttt{var1} is equal to
999999. Can be replaced with any other values or logical expressions. \\
\bottomrule
\end{longtable}

\hypertarget{creating-binary-variables}{%
\subsection*{Creating binary
variables}\label{creating-binary-variables}}

Recall that there are two easy ways to make binary variables out of
categorical or continuous variables. Consider the variable
\texttt{race}, where 100 = White, 200 = Black, 300 = Native American,
651 = Asian, etc. Suppose you want to generate a binary indicator for
whether a person is White.

\begin{itemize}
\item
  \texttt{gen\ white\ =\ race\ ==\ 100}: generates a variable equal to 1
  if \texttt{race} is 100, and 0 otherwise
\item
  \texttt{gen\ white\ =\ 1\ if\ race\ ==\ 100}: generates a variable
  equal to 1 if \texttt{race} is 100, and \textbf{missing} otherwise. To
  complete this you need two lines of code:\\
  \texttt{gen\ white\ =\ 1\ if\ race\ ==\ 100}~\\
  \texttt{replace\ white\ =\ 0\ if\ race\ !=\ 100}
\end{itemize}

\hypertarget{working-with-loops}{%
\subsection*{Working with loops}\label{working-with-loops}}

Loops can help us (1) avoid errors and (2) code super fast!

I've uploaded a sample from our class
\href{https://ec200f21.netlify.app/assignment/materials/lab4_sample.do}{\texttt{here,\ as\ lab4\_sample.do}}

Stata has two types of looping setups, using the \texttt{forval} or
\texttt{foreach} command. The first is simpler, and the second is more
versatile. Recall that you can always use \texttt{help\ forval} or
\texttt{help\ foreach} if your code isn't working or if you have a
vision you're not sure how to realize.

\hypertarget{looping-with-forval}{%
\paragraph{\texorpdfstring{Looping with
\texttt{forval}}{Looping with forval}}\label{looping-with-forval}}

\begin{verbatim}
forvalues lname = range {
                Stata commands referring to `lname'
        }
\end{verbatim}

What does each component mean?

\begin{itemize}
\tightlist
\item
  \texttt{forvalues}: this is the command. You can abbreviate it as
  \texttt{forval}.
\item
  \texttt{lname}: this is a variable you make up. Often, people will
  just use \texttt{i}, becuase we're just counting. It will take on the
  values in \texttt{range} as it increments through the loop. It is a
  \textbf{local} variable, meaning that you have to call it using
  `lname', and not as lname (need those punctuation marks!) and that it
  is only saved as long as your do-file is running.\\
\item
  \texttt{range}: this is the set of values that the local variable will
  iterate over
\item
  Brackets: Open bracket needs to be on same line as the \texttt{forval}
  command. Close bracket needs to be on its own line.
\end{itemize}

\begin{verbatim}
forval i = 0/2{
gen labfor`i' = labforce == `i'
}
\end{verbatim}

What does this do? It creates a loop for which local variable
`\texttt{i\textquotesingle{}} is first 0, then 1, then 2. Within the
loop, it generates \texttt{labfor0}, which is equal to 1 if
\texttt{labforce} equals 0 (not in universe), it generates
\texttt{labfor1}, which is equal to 1 if \texttt{labforce} equals 1 (not
in labor force), and it generates \texttt{labfor2}, which is equal to 1
if \texttt{labforce} equals 2 (in labor force).

The choice of ranges can be done in other ways:

\begin{itemize}
\tightlist
\item
  \texttt{forval\ i\ =\ 0/10}: hits every integer between 0 and 10 - 0,
  1, 2, \ldots{} 10
\item
  \texttt{forval\ i\ =\ 1(10)100}: starts at 1, then increments by 10,
  stopping at 100: 1, 11, 21, 31, \ldots{} 91
\item
  \texttt{forvalues\ k\ =\ 5\ 10\ to\ 300}: starts at 10, then
  increments by 5 until 300: 5, 10, 15, \ldots{}
\end{itemize}

See \texttt{help\ forval} for more options

\hypertarget{looping-with-foreach}{%
\paragraph{\texorpdfstring{Looping with
\texttt{foreach}}{Looping with foreach}}\label{looping-with-foreach}}

This command lets you loop through number lists (like above), but also
through sets of variables, values, names, etc. You can approach it two
ways:

\begin{itemize}
\tightlist
\item
  Do not specify the type of list, use \textbf{in}:
  \texttt{foreach\ lname\ in\ list}:
\item
  Specify the type of list (\texttt{listtype}), use \textbf{of}:
  \texttt{foreach\ lname\ of\ listtype\ list}
\end{itemize}

This is confusing until we see examples:

\begin{verbatim}
foreach x in "rice wheat corn rye barley oats" {
          display "`x'"
        }
\end{verbatim}

This will start with \texttt{x} equal to the string ``rice''. Then, it
will run with \texttt{x} equal to ``wheat'', etc.

\begin{verbatim}
    foreach num of numlist 1 4/8 13(2)21 103 {
        display `num'
 }
\end{verbatim}

This will loop over 1, 4, 5, 6, 7, 8, 13, 15, 17, \ldots{}

You can loop over variable names too!

\begin{verbatim}
foreach var of varlist inc* {
      summarize `var',d
        }
\end{verbatim}

This summarizes (with detail) each variable that starts with
\texttt{inc}

\hypertarget{working-with-binary-independent-variables}{%
\subsection*{Working with binary independent
variables}\label{working-with-binary-independent-variables}}

When you are representing a categorical variable with a set of binary
variables, there is a slow way and a fast way to integrate them.

\begin{itemize}
\tightlist
\item
  Slow way: generate the binary variables you want, and include them.
  This is good when you want to be precise about your omitted variable,
  or when you want to create complicated binary categories
\end{itemize}

\begin{verbatim}
gen white_nh = race == 100 & hisp == 0 
gen black_nh = race == 200 & hisp == 1
gen hisp = hisp == 1
gen other = white_nh == 0 & black_nh == 0 & hisp == 0 
regress incwage black_nh hisp other
\end{verbatim}

Here, white, non-Hispanic is the omitted ``reference'' category.

\begin{itemize}
\tightlist
\item
  Fast way: tell Stata to create a binary variable for each value of a
  categorical variable. This is good when you aren't trying to do
  anything complicated and when you want to be quick - very useful if
  you want something like state-level dummies.
\end{itemize}

\begin{verbatim}
regress incwage i.statefip
\end{verbatim}

Note that this will work only if your categorical variable is numeric.
If it's a string you'll get an error. You can fix it by adding a
\texttt{xi:} prefix, like so:

\begin{verbatim}
xi: regress incwage i.statefip
\end{verbatim}

When we include a dummy variable for every value of a categorical
variable, like above, we call those ``fixed effects.'' We'll talk about
these more soon.

\hypertarget{reading-regression-tables-reminder}{%
\section*{Reading regression tables
(reminder!)}\label{reading-regression-tables-reminder}}

\hypertarget{lab-4-worksheet}{%
\subsection*{Lab 4 Worksheet}\label{lab-4-worksheet}}
\addcontentsline{toc}{subsection}{Lab 4 Worksheet}

Download the do-file template and data files. Personalize the file paths
so that you can run it and open your \texttt{cps\_2016.dta} file. You
can also work with a blank data file if you're more comfortable - just
make sure you remember to include commands to start and close your log
file.

\emph{Use robust standard errors in all regressions}

\begin{enumerate}
\def\labelenumi{\arabic{enumi}.}
\item
  Let's practice with loops! Download
  \href{https://ec200f21.netlify.app/assignment/materials/loop_example.do}{loop\_example.do} and paste the
  code into your sample. Run it and look at the output. In your do-file,
  write comments that describe what each loop is going.
\item
  Now, go back to your \texttt{cps\_2016.dta} file and do-file template.
  Adjust your do-file template so that it loads \texttt{cps\_2016.dta}
  and starts a log.
\item
  Restrict your sample to individuals ages 25-54.
\item
  Create a new variable, \texttt{birthyr}, equal to each individual's
  year of birth. Is there any potential imprecision or error in this
  variable?
\item
  Then, write a loop to generate a dummy variable for each possible
  value of birth year.\footnote{There is a faster way to do this, using
    \texttt{xi\ i.birthyr}, but we're learning about loops, so just go
    with it.}
\end{enumerate}

\begin{enumerate}
\def\labelenumi{\arabic{enumi}.}
\setcounter{enumi}{5}
\item
  Look through the available list of data (note,
  \href{https://cps.ipums.org/cps/}{IPUMS} has full documentation of all
  variables). Based on this data, think of a research question for your
  lab of the form, ``What is the relationship between .\ldots{} and
  ...?''. Pick a dependent variable that is \textbf{continuous}.
  \emph{(Because a later question asks you to explore race/ethnicity
  controls, please do not use a race/ethnicity variable for \(X\).)}

  Research question:\\

  Dependent variable (\(Y\)):\\

  Key independent variable (\(X\)):\\
\item
  Using the data available, write a reasonable population model,
  including your key independent variable along with a set of likely
  relevant independent variables (somewhere between 2 and 5 additional
  variables). Before estimating your regression, you should tabulate
  each variable to make sure you are interpreting it correctly.
\item
  In words, what exactly will your estimated regression tell us?
\item
  What do you hypothesize the answer to your research question is?
  (i.e.~strong positive, weak negative, none)
\item
  Before you estimate your model, make sure you don't have any N/A
  values coded. For example, if \texttt{incwage} is not applicable, it
  is coded as \texttt{9999999}. Tabulate or summarize your data to check
  for any values like this. Replace any values as missing if they are
  equal to some N/A code (see above).
\item
  Estimate the relationship between \(X\) and \(Y\) using simple linear
  regression (excluding any other covariates). Write your results in
  equation form and report the \(R^2\). How many degrees of freedom do
  you have?
\item
  Estimate the relationship between \(X\) and \(Y\) using multiple
  linear regression (including other covariates). That is, estimate the
  population model you wrote earlier. Write your results in equation
  form and report the \(R^2\). How many degrees of freedom do you have?
\item
  Using your multivariate linear regression from the previous step, set
  up a hypothesis test for your parameter of interest, the \(\beta\)
  associated with your key independent variable, \(X\). What do you
  find? What is the p-value? What is the interpretation?
\item
  Besides your key independent variable, which other variables are
  statistically significant at the five-percent level?
\item
  A lot of student research papers will look at differences in outcomes
  by gender and by racial/ethnic groups. U.S. surveys like the CPS, ACS,
  and Census treat race and ethnicity a little strangely, and it can
  take some practice to get comfortable.

  There are two variables commonly used to identify a person's race and
  ethnicity: the \texttt{race} and the \texttt{hispan} variable.

  \begin{enumerate}
  \def\labelenumii{\arabic{enumii}.}
  \item
    What share of the sample is White, non-Hispanic?
  \item
    What share of the sample is Hispanic/Latino?
  \item
    A common way to summarize the racial/ethnic make-up of the U.S. is
    the following categories:

    \begin{itemize}
    \item
      White, non-Hispanic
    \item
      Black, non-Hispanic
    \item
      Hispanic/Latino
    \item
      Asian, non-Hispanic
    \item
      Other
    \end{itemize}

    Make a table that shows the distribution of the population into
    these five groupings.
  \end{enumerate}
\item
  Estimate your multiple linear regression model from earlier, but
  include the race/ethnicity variables that you created in the previous
  part. How do the inclusion of these factors affect your estimates of
  the relationship between \(Y\) and \(X\)?
\item
  Now, add ``birth-year fixed effects'' to your regression that you
  generated earlier. Because there is a set of binary 0/1 variables, one
  for each year of birth, they will essentially pull out any mean
  differences in your dependent variable at the birth-year level - so if
  your outcome variable is different for people born in 1971 vs 1971 on
  average, these variables will take care of it. What is the omitted
  birth cohort? How do the inclusion of these factors affect your
  estimates of the relationship between \(Y\) and \(X\)?
\end{enumerate}

\end{document}
