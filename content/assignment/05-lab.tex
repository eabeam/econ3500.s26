\documentclass[11pt]{article}




\usepackage[sfdefault]{FiraSans} %% option 'sfdefault' activates Fira Sans as the default text font
\usepackage[T1]{fontenc}
\renewcommand*\oldstylenums[1]{{\firaoldstyle #1}}

\usepackage{natbib}
\usepackage[french,english]{babel}
\usepackage{numprint}
\usepackage{multirow}
\usepackage{rotating}
\usepackage{fancyhdr}
\usepackage{booktabs}
\usepackage{multicol}
\usepackage{hyperref}\hypersetup{colorlinks=true}

\usepackage{amsmath,amssymb,amsfonts,textcomp}
\usepackage{color}
\usepackage{calc}
 \setlength{\tabcolsep}{8pt}
\usepackage{setspace}
\onehalfspacing
\usepackage{longtable}
\usepackage{graphicx}
\usepackage[margin=1in]{geometry}
\setlength{\parindent}{0pt}
\usepackage[bottom]{footmisc}
\pagestyle{fancy}
\usepackage{titlesec}
\usepackage{lipsum}
\usepackage{cancel}
\usepackage{multicol}

\usepackage{amsmath,amssymb}
\usepackage{lmodern}
\usepackage{iftex}
\ifPDFTeX
  \usepackage[T1]{fontenc}
  \usepackage[utf8]{inputenc}
  \usepackage{textcomp} % provide euro and other symbols
\else % if luatex or xetex
  \usepackage{unicode-math}
  \defaultfontfeatures{Scale=MatchLowercase}
  \defaultfontfeatures[\rmfamily]{Ligatures=TeX,Scale=1}
\fi

\titleformat{\section}
  {\normalfont\Large\scshape\bfseries}{\thesection}{1em}{}
  \titlespacing{\section}{0pt}{10pt}{0pt}
\titleformat{\subsection}
  {\normalfont\bfseries}{\thesection}{1em}{}
  \titlespacing{\subsection}{0pt}{6pt}{0pt}
\providecommand{\tightlist}{%
  \setlength{\itemsep}{0pt}\setlength{\parskip}{0pt}}\newenvironment{itemize*}%

  
\lhead{EC200 - Econometrics and Applications}
\rhead{Version: \today}
\setlength\parskip{0.10in}
\begin{document}
\thispagestyle{plain}
\singlespacing


Version: \today \hfill Spring 2022\\
EC200: Econometrics and Applications
%\vspace{1.5cm}
\begin{center}
\Large{\textbf{Lab 5: Merging and hypothesis tests}}\\
\end{center}
\bigskip


%{
%\setcounter{tocdepth}{3}
%\tableofcontents
%}

\hypertarget{objectives}{%
\subsection*{Objectives}\label{objectives}}
\addcontentsline{toc}{subsection}{Objectives}

Today we're going to keep working with
\href{../materials/cps_2016.dta}{\texttt{cps\_2016.dta}}, which contains
information from the \href{https://cps.ipums.org/cps/}{2016 Current
Population Survey}.

We're going to merge in county-level unemployment rates from the
\href{https://www.bls.gov/lau/tables.htm}{BLS}

By the end of this lab, you should be able to complete the following
tasks in Stata:

\begin{itemize}
\item
  Import data from Excel
\item
  Merge data sets
\item
  Test hypotheses for linear combinations of coefficients
\item
  Test the general significance of a regression
\end{itemize}

\hypertarget{key-commands}{%
\subsection*{Key commands}\label{key-commands}}
\addcontentsline{toc}{subsection}{Key commands}

\begin{longtable}[]{@{}
  >{\raggedright\arraybackslash}p{(\columnwidth - 2\tabcolsep) * \real{0.56}}
  >{\raggedleft\arraybackslash}p{(\columnwidth - 2\tabcolsep) * \real{0.44}}@{}}
\toprule
command & description \\
\midrule
\endhead
Importing data & \\
\texttt{import\ excel\ using\ “file1.xlsx”,\ firstrow\ clear} & Import
the data set file1.xlsx from excel into Stata. The \texttt{firstrow}
option tells Stata to use the first row as the variable name. The
\texttt{clear} option tells Stata to erase any data already in the
set \\
Identifying duplicates & \\
\texttt{duplicates\ list\ var1\ var2} & List any observations that are
duplicates on the listed variables, \emph{var1} \texttt{var2}, etc. \\
\texttt{duplicates\ tag\ var1\ var2,\ gen(d1)} & Generate a new
variable, \texttt{d1} that indicates which variables are duplicates for
\texttt{var1} and \texttt{var2} \\
Merging datasets & \\
\texttt{merge\ 1:1\ var1\ var2\ using\ file2} & Perform a one-to-one
merge based on \texttt{var1} and \texttt{var2}. There cannot be any
duplicates on the variables you are using to merge \\
\texttt{merge\ m:1\ var1\ var2\ using\ file2} & Perform a many-to-one
merge based on \texttt{var1} and \texttt{var2}. There can be duplicate
identifiers in the master data set (like if merging state data to
individuals), but there should be no duplicates in the using data set \\
Converting between string and numeric variables & \\
\texttt{decode\ var1,\ gen(newvar)} & Take a numeric variable with
labels and generate a new string variable that is equal to the values of
those labels. (You can do the opposite with \texttt{encode}). \\
\texttt{destring\ var1,\ replace} & Take a string variable,
\texttt{var1} and convert it to a numeric variable, replacing the old
variable \\
\texttt{tostring\ var2,\ gen(string\_var)} & Take a numeric variable,
\texttt{var2} and make it a string, but make that into a new variable
called \texttt{string\_var} \\
Statistical tests & \\
\texttt{test\ var1\ =\ var2} & Run after estimating a regression. Tests
the null hypothesis that the coefficient on \texttt{var1} equals the
coefficient on \texttt{var2}, against the two-sided alternative. \\
\texttt{testparm\ var1\ var2\ ...} & Run after estimating a regression.
Tests the whether all listed variables, \texttt{var1} , \texttt{var2},
etc., are jointly equal to zero, against the two-sided alternative. \\
\bottomrule
\end{longtable}

\hypertarget{a-note-on-temporary-files-optional}{%
\subsubsection{A note on temporary files
(optional)}\label{a-note-on-temporary-files-optional}}

This exercise works by having two data sets stored on your hard drive,
then running a \texttt{merge} command to unite them. You might notice
that the workflow feels clunky and generates extra files - open a data
set, save it, open another data set, then merge in the first data set.

You can use temporary files to speed things up! Basically, you can save
files in your local memory, and call those files the same way we called
local variables. Everything has to be run in the do-file for this to
work.

A short example (you can paste this in a do-file and run it, as it uses
Stata files) :

\begin{verbatim}
tempfile tempauto       // Declare tempfile (needs to run before you try to save)

webuse autosize,clear

save `tempauto', replace    // save to temp file t1

webuse autoexpense, clear 

merge 1:1 make using `tempauto'   // call tempfile

tab _merge    // check out merge

list
\end{verbatim}

\texttt{r\ htmltools::HTML("\{\{\textless{}\ youtube\ 8yfXvk8QYy0\ \textgreater{}\}\}")}

\hypertarget{lab-5-worksheet}{%
\subsection{Lab 5 Worksheet}\label{lab-5-worksheet}}

\hypertarget{what-do-i-submit}{%
\subsubsection{What do I submit?}\label{what-do-i-submit}}

\begin{itemize}
\tightlist
\item
  Your written up answers to the exercise questions. This can be typed
  or written out then scanned (or photographed), in any reasonable
  format.
\item
  The do-file you've created that runs this analysis
\item
  A log file that contains the results from this exercise.
\end{itemize}

\hypertarget{exercises}{%
\subsubsection{Exercises}\label{exercises}}

\begin{enumerate}
\def\labelenumi{\arabic{enumi}.}
\item
  Visit \url{https://www.bls.gov/lau/tables.htm} to access 2016 annual
  \textbf{county-level} \emph{annual} unemployment rates.

  \begin{enumerate}
  \def\labelenumii{\arabic{enumii}.}
  \item
    Download the appropriate table.
  \item
    Rename variables as needed, and delete any unnecessary cells. If you
    want your life to be easier, make the first row include your
    variable names, and then have the data start in second
    row.\footnote{You can also sort this out w/ Stata commands if you'd
      rather work with the raw, unedited file}
  \item
    Save your revised file.
  \end{enumerate}
\item
  Open Stata, start a new do-file (or bring in a template). Make sure
  you add code to start (and end) a log.
\item
  Import your unemployment excel into Stata and save it as a data file,
  \texttt{unemp.dta}.
\item
  Open \texttt{cps\_2016.dta} and restrict the sample to adults (age
  18+).
\item
  Now, merge your unemployment data into \texttt{cps\_2016.dta} by
  county. This may not be smooth. A few tips:

  \begin{enumerate}
  \def\labelenumii{\arabic{enumii}.}
  \item
    The FIPS codes are in different formats between the two data sets. A
    county code like this ``55083'' contins a state part (55) and a
    county part (083).
  \item
    You can convert a variable to and from a string using the commands
    \texttt{destring\ var1,replace} and \texttt{tostring\ var2,replace},
    respecitvely.
  \item
    You can concatenate string variables by adding them like this:
    \texttt{gen\ newvar\ =\ string1\ +\ string2}
  \item
    Determine whether you need a one-to-one or many-to-one merge.
  \item
    You may get errors, and you'll need to fix these to have a
    successful merge.
  \end{enumerate}
\item
  You've done it! Tabulate the new variable \texttt{\_merge}. What share
  of observations successfully merge?\footnote{To get a sense if you've
    done this right, about 40-45\% of observations should match. This is
    because the CPS will withhold county-level identifiers for very
    small counties to protect confidentiality.}
\item
  Drop any unmatched observations (you can use \texttt{drop\ if}, as
  we'll retain this restriction for the rest of the exercise.) What is
  the average unemployment rate for the entire sample - why would this
  be different than taking the average of county-level unemployment
  rates in your excel file?
\item
  Why can't we use education as a linear variable?
\item
  Generate three dummy variables. These three variables should be
  mutually exclusive, and they should not be missing for any people.

  \begin{itemize}
  \item
    \texttt{lesshs}, a variable equal to one if a person completed
    \emph{less than} a high school diploma
  \item
    \texttt{hsgrad}, a variable equal to one if a person completed at
    least a high school but less than a Bachelor's degree
  \item
    \texttt{colgrad}, a variable equal to one if a person completed a
    Bachelor's degree or higher
  \end{itemize}

  \emph{Note:} Education is coded with \textbf{labels,} which means that
  it is numeric data with a description of what each number means on
  top. These show up as blue in the Stata browser. To view variables
  without the labels, add the no-label
  option:\texttt{tab\ educ,\ nolabel}.
\item
  What is the mean of \texttt{lesshs}, \texttt{hsgrad}, and
  \texttt{colgrad}?
\item
  Estimate a regression of total personal income on education, usingthe
  binary variables you just created. Omit \texttt{lesshs}.
\item
  Set up a hypothesis test for whether both \texttt{hsgrad} and
  \texttt{colgrad} are jointly significant. Report the null hypothesis,
  alternative hypothesis, test statistic, and conclusion.
\item
  Set up a hypothesis test for whether the returns to being a
  high-school graduate are the same as the returns to being a college
  graduate. Report the null hypothesis, alternative hypothesis, test
  statistic, and conclusion.
\item
  Is this regression significant overall? Explain how you know.
\item
  Now add county-level unemployment rate to the previous equation. What
  is the interpretation of the coefficient on unemployment? Is it
  statistically significant?
\item
  Estimate the same equation by regressing total personal income on the
  education binary variables and county-level unemployment, restricting
  to those who are currently in the labor force. How does this change
  the coefficient on unemployment?
\item
  Identify three \emph{state} or \emph{county-level} variables that are
  likely to cause omitted variable bias if you want to know whether
  unemployment affects individual wages.
\item
  For \emph{one} of the variables you listed above, find the data
  online, import into Stata, and merge it in.
\item
  Regress total personal income on the education binary variables,
  county-level unemployment, and the new variable you found. Restrict
  your sample to those who are currently in the labor force. How does
  the inclusion of your new variable affect the coefficient on
  unemployment?
\end{enumerate}

\end{document}
