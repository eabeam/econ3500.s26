% Options for packages loaded elsewhere
\PassOptionsToPackage{unicode}{hyperref}
\PassOptionsToPackage{hyphens}{url}
%
\documentclass[
]{article}
\usepackage{lmodern}
\usepackage{amssymb,amsmath}
\usepackage{ifxetex,ifluatex}
\ifnum 0\ifxetex 1\fi\ifluatex 1\fi=0 % if pdftex
  \usepackage[T1]{fontenc}
  \usepackage[utf8]{inputenc}
  \usepackage{textcomp} % provide euro and other symbols
\else % if luatex or xetex
  \usepackage{unicode-math}
  \defaultfontfeatures{Scale=MatchLowercase}
  \defaultfontfeatures[\rmfamily]{Ligatures=TeX,Scale=1}
\fi
% Use upquote if available, for straight quotes in verbatim environments
\IfFileExists{upquote.sty}{\usepackage{upquote}}{}
\IfFileExists{microtype.sty}{% use microtype if available
  \usepackage[]{microtype}
  \UseMicrotypeSet[protrusion]{basicmath} % disable protrusion for tt fonts
}{}
\makeatletter
\@ifundefined{KOMAClassName}{% if non-KOMA class
  \IfFileExists{parskip.sty}{%
    \usepackage{parskip}
  }{% else
    \setlength{\parindent}{0pt}
    \setlength{\parskip}{6pt plus 2pt minus 1pt}}
}{% if KOMA class
  \KOMAoptions{parskip=half}}
\makeatother
\usepackage{xcolor}
\IfFileExists{xurl.sty}{\usepackage{xurl}}{} % add URL line breaks if available
\IfFileExists{bookmark.sty}{\usepackage{bookmark}}{\usepackage{hyperref}}
\hypersetup{
  pdftitle={Lab 5: Merging and hypothesis tests},
  hidelinks,
  pdfcreator={LaTeX via pandoc}}
\urlstyle{same} % disable monospaced font for URLs
\usepackage{longtable,booktabs}
% Correct order of tables after \paragraph or \subparagraph
\usepackage{etoolbox}
\makeatletter
\patchcmd\longtable{\par}{\if@noskipsec\mbox{}\fi\par}{}{}
\makeatother
% Allow footnotes in longtable head/foot
\IfFileExists{footnotehyper.sty}{\usepackage{footnotehyper}}{\usepackage{footnote}}
\makesavenoteenv{longtable}
\setlength{\emergencystretch}{3em} % prevent overfull lines
\providecommand{\tightlist}{%
  \setlength{\itemsep}{0pt}\setlength{\parskip}{0pt}}
\setcounter{secnumdepth}{-\maxdimen} % remove section numbering
\ifluatex
  \usepackage{selnolig}  % disable illegal ligatures
\fi

\title{Lab 5: Merging and hypothesis tests}
\author{}
\date{2020-10-14}

\begin{document}
\maketitle



\hypertarget{objectives}{%
\subsection*{Objectives}\label{objectives}}
\addcontentsline{toc}{subsection}{Objectives}

Today we're going to keep working with
\href{../materials/cps_2016.dta}{\texttt{cps\_2016.dta}}, which contains
information from the \href{https://cps.ipums.org/cps/}{2016 Current
Population Survey}. We're going to merge in county-level unemployment
rates from the \href{https://www.bls.gov/lau/tables.htm}{BLS}

By the end of this lab, you should be able to complete the following
tasks in Stata:

\begin{itemize}
\item
  Import data from Excel
\item
  Merge data sets
\item
  Test hypotheses for linear combinations of coefficients
\item
  Test the general significance of a regression
\end{itemize}

\hypertarget{key-commands}{%
\subsection*{Key commands}\label{key-commands}}
\addcontentsline{toc}{subsection}{Key commands}

\begin{longtable}[]{@{}lr@{}}
\toprule
\begin{minipage}[b]{0.53\columnwidth}\raggedright
command\strut
\end{minipage} & \begin{minipage}[b]{0.41\columnwidth}\raggedleft
description\strut
\end{minipage}\tabularnewline
\midrule
\endhead
\begin{minipage}[t]{0.53\columnwidth}\raggedright
Importing data\strut
\end{minipage} & \begin{minipage}[t]{0.41\columnwidth}\raggedleft
\strut
\end{minipage}\tabularnewline
\begin{minipage}[t]{0.53\columnwidth}\raggedright
\texttt{import\ excel\ using\ “file1.xlsx”,\ firstrow\ clear}\strut
\end{minipage} & \begin{minipage}[t]{0.41\columnwidth}\raggedleft
Import the data set file1.xlsx from excel into Stata. The
\texttt{firstrow} option tells Stata to use the first row as the
variable name. The \texttt{clear} option tells Stata to erase any data
already in the set\strut
\end{minipage}\tabularnewline
\begin{minipage}[t]{0.53\columnwidth}\raggedright
Identifying duplicates\strut
\end{minipage} & \begin{minipage}[t]{0.41\columnwidth}\raggedleft
\strut
\end{minipage}\tabularnewline
\begin{minipage}[t]{0.53\columnwidth}\raggedright
\texttt{duplicates\ list\ var1\ var2}\strut
\end{minipage} & \begin{minipage}[t]{0.41\columnwidth}\raggedleft
List any observations that are duplicates on the listed variables,
\emph{var1} \texttt{var2}, etc.\strut
\end{minipage}\tabularnewline
\begin{minipage}[t]{0.53\columnwidth}\raggedright
\texttt{duplicates\ tag\ var1\ var2,\ gen(d1)}\strut
\end{minipage} & \begin{minipage}[t]{0.41\columnwidth}\raggedleft
Generate a new variable, \texttt{d1} that indicates which variables are
duplicates for \texttt{var1} and \texttt{var2}\strut
\end{minipage}\tabularnewline
\begin{minipage}[t]{0.53\columnwidth}\raggedright
Merging datasets\strut
\end{minipage} & \begin{minipage}[t]{0.41\columnwidth}\raggedleft
\strut
\end{minipage}\tabularnewline
\begin{minipage}[t]{0.53\columnwidth}\raggedright
\texttt{merge\ 1:1\ var1\ var2\ using\ file2}\strut
\end{minipage} & \begin{minipage}[t]{0.41\columnwidth}\raggedleft
Perform a one-to-one merge based on \texttt{var1} and \texttt{var2}.
There cannot be any duplicates on the variables you are using to
merge\strut
\end{minipage}\tabularnewline
\begin{minipage}[t]{0.53\columnwidth}\raggedright
\texttt{merge\ m:1\ var1\ var2\ using\ file2}\strut
\end{minipage} & \begin{minipage}[t]{0.41\columnwidth}\raggedleft
Perform a many-to-one merge based on \texttt{var1} and \texttt{var2}.
There can be duplicate identifiers in the master data set (like if
merging state data to individuals), but there should be no duplicates in
the using data set\strut
\end{minipage}\tabularnewline
\begin{minipage}[t]{0.53\columnwidth}\raggedright
Converting between string and numeric variables\strut
\end{minipage} & \begin{minipage}[t]{0.41\columnwidth}\raggedleft
\strut
\end{minipage}\tabularnewline
\begin{minipage}[t]{0.53\columnwidth}\raggedright
\texttt{decode\ var1,\ gen(newvar)}\strut
\end{minipage} & \begin{minipage}[t]{0.41\columnwidth}\raggedleft
Take a numeric variable with labels and generate a new string variable
that is equal to the values of those labels. (You can do the opposite
with \texttt{encode}).\strut
\end{minipage}\tabularnewline
\begin{minipage}[t]{0.53\columnwidth}\raggedright
\texttt{destring\ var1,\ replace}\strut
\end{minipage} & \begin{minipage}[t]{0.41\columnwidth}\raggedleft
Take a string variable, \texttt{var1} and convert it to a numeric
variable, replacing the old variable\strut
\end{minipage}\tabularnewline
\begin{minipage}[t]{0.53\columnwidth}\raggedright
\texttt{tostring\ var2,\ gen(string\_var)}\strut
\end{minipage} & \begin{minipage}[t]{0.41\columnwidth}\raggedleft
Take a numeric variable, \texttt{var2} and make it a string, but make
that into a new variable called \texttt{string\_var}\strut
\end{minipage}\tabularnewline
\begin{minipage}[t]{0.53\columnwidth}\raggedright
Statistical tests\strut
\end{minipage} & \begin{minipage}[t]{0.41\columnwidth}\raggedleft
\strut
\end{minipage}\tabularnewline
\begin{minipage}[t]{0.53\columnwidth}\raggedright
\texttt{test\ var1\ =\ var2}\strut
\end{minipage} & \begin{minipage}[t]{0.41\columnwidth}\raggedleft
Run after estimating a regression. Tests the null hypothesis that the
coefficient on \texttt{var1} equals the coefficient on \texttt{var2},
against the two-sided alternative.\strut
\end{minipage}\tabularnewline
\begin{minipage}[t]{0.53\columnwidth}\raggedright
\texttt{testparm\ var1\ var2\ ...}\strut
\end{minipage} & \begin{minipage}[t]{0.41\columnwidth}\raggedleft
Run after estimating a regression. Tests the whether all listed
variables, \texttt{var1} , \texttt{var2}, etc., are jointly equal to
zero, against the two-sided alternative.\strut
\end{minipage}\tabularnewline
\bottomrule
\end{longtable}

\hypertarget{a-note-on-temporary-files-optional}{%
\subsubsection{A note on temporary files
(optional)}\label{a-note-on-temporary-files-optional}}

This exercise works by having two data sets stored on your hard drive,
then running a \texttt{merge} command to unite them. You might notice
that the workflow feels clunky and generates extra files - open a data
set, save it, open another data set, then merge in the first data set.

You can use temporary files to speed things up! Basically, you can save
files in your local memory, and call those files the same way we called
local variables. Everything has to be run in the do-file for this to
work.

A short example (you can paste this in a do-file and run it, as it uses
Stata files) :

\begin{verbatim}
tempfile tempauto       // Declare tempfile (needs to run before you try to save)

webuse autosize,clear

save `tempauto', replace    // save to temp file t1

webuse autoexpense, clear 

merge 1:1 make using `tempauto'   // call tempfile

tab _merge    // check out merge

list
\end{verbatim}

\hypertarget{exercises}{%
\subsubsection{Exercises}\label{exercises}}

\begin{enumerate}
\def\labelenumi{\arabic{enumi}.}
\item
  Visit \url{https://www.bls.gov/lau/tables.htm} to access 2016 annual
  \textbf{county-level} \emph{annual} unemployment rates.

  \begin{enumerate}
  \def\labelenumii{\arabic{enumii}.}
  \item
    Download the appropriate table.
  \item
    Rename variables as needed, and delete any unnecessary cells. If you
    want your life to be easier, make the first row include your
    variable names, and then have the data start in second
    row.\footnote{You can also sort this out w/ Stata commands if you'd
      rather work with the raw, unedited file}
  \item
    Save your revised file.
  \end{enumerate}
\item
  Open Stata, start a new do-file (or bring in a template). Make sure
  you add code to start (and end) a log.
\item
  Import your unemployment excel into Stata and save it as a data file,
  \texttt{unemp.dta}.
\item
  Open \texttt{cps\_2016.dta} and restrict the sample to adults (age
  18+).
\item
  Now, merge your unemployment data into \texttt{cps\_2016.dta} by
  county. This may not be smooth. A few tips:

  \begin{enumerate}
  \def\labelenumii{\arabic{enumii}.}
  \item
    The FIPS codes are in different formats between the two data sets. A
    county code like this ``55083'' contins a state part (55) and a
    county part (083).
  \item
    You can convert a variable to and from a string using the commands
    \texttt{destring\ var1,replace} and \texttt{tostring\ var2,replace},
    respecitvely.
  \item
    You can concatenate string variables by adding them like this:
    \texttt{gen\ newvar\ =\ string1\ +\ string2}
  \item
    Determine whether you need a one-to-one or many-to-one merge.
  \item
    You may get errors, and you'll need to fix these to have a
    successful merge.
  \end{enumerate}
\item
  You've done it! Tabulate the new variable \texttt{\_merge}. What share
  of observations successfully merge?\footnote{To get a sense if you've
    done this right, about 40-45\% of observations should match. This is
    because the CPS will withhold county-level identifiers for very
    small counties to protect confidentiality.}
\item
  Drop any unmatched observations (you can use \texttt{drop\ if}, as
  we'll retain this restriction for the rest of the exercise.) What is
  the average unemployment rate for the entire sample - why would this
  be different than taking the average of county-level unemployment
  rates in your excel file?
\item
  Why can't we use education as a linear variable?
\item
  Generate three dummy variables. These three variables should be
  mutually exclusive, and they should not be missing for any people.

  \begin{itemize}
  \item
    \texttt{lesshs}, a variable equal to one if a person completed
    \emph{less than} a high school diploma
  \item
    \texttt{hsgrad}, a variable equal to one if a person completed at
    least a high school but less than a Bachelor's degree
  \item
    \texttt{colgrad}, a variable equal to one if a person completed a
    Bachelor's degree or higher
  \end{itemize}

  \emph{Note:} Education is coded with \textbf{labels,} which means that
  it is numeric data with a description of what each number means on
  top. These show up as blue in the Stata browser. To view variables
  without the labels, add the no-label
  option:\texttt{tab\ educ,\ nolabel}.
\item
  What is the mean of \texttt{lesshs}, \texttt{hsgrad}, and
  \texttt{colgrad}?
\item
  Estimate a regression of total personal income on education, usingthe
  binary variables you just created. Omit \texttt{lesshs}.
\item
  Set up a hypothesis test for whether both \texttt{hsgrad} and
  \texttt{colgrad} are jointly significant. Report the null hypothesis,
  alternative hypothesis, test statistic, and conclusion.
\item
  Set up a hypothesis test for whether the returns to being a
  high-school graduate are the same as the returns to being a college
  graduate. Report the null hypothesis, alternative hypothesis, test
  statistic, and conclusion.
\item
  Is this regression significant overall? Explain how you know.
\item
  Now add county-level unemployment rate to the previous equation. What
  is the interpretation of the coefficient on unemployment? Is it
  statistically significant?
\item
  Estimate the same equation by regressing total personal income on the
  education binary variables and state-level unemployment, restricting
  to those who are currently in the labor force. How does this change
  the coefficient on unemployment?
\item
  Identify three \emph{state-level} variables that are likely to cause
  omitted variable bias if you want to know whether unemployment affects
  individual wages.
\item
  For \emph{one} of the variables you listed above, find the data
  online, import into Stata, and merge it in.
\item
  Regress total personal income on the education binary variables,
  state-level unemployment, and the new variable you found. Restrict
  your sample to those who are currently in the labor force. How does
  the inclusion of your new variable affect the coefficient on
  unemployment?
\end{enumerate}

\end{document}
