\documentclass[11pt]{article}




\usepackage[sfdefault]{FiraSans} %% option 'sfdefault' activates Fira Sans as the default text font
\usepackage[T1]{fontenc}
\renewcommand*\oldstylenums[1]{{\firaoldstyle #1}}

\usepackage{natbib}
\usepackage[french,english]{babel}
\usepackage{numprint}
\usepackage{multirow}
\usepackage{rotating}
\usepackage{fancyhdr}
\usepackage{booktabs}
\usepackage{multicol}
\usepackage{hyperref}\hypersetup{colorlinks=true}

\usepackage{amsmath,amssymb,amsfonts,textcomp}
\usepackage{color}
\usepackage{calc}
 \setlength{\tabcolsep}{8pt}
\usepackage{setspace}
\onehalfspacing
\usepackage{longtable}
\usepackage{graphicx}
\usepackage[margin=1in]{geometry}
\setlength{\parindent}{0pt}
\usepackage[bottom]{footmisc}
\pagestyle{fancy}
\usepackage{titlesec}
\usepackage{lipsum}
\usepackage{cancel}
\usepackage{multicol}

\usepackage{amsmath,amssymb}
\usepackage{lmodern}
\usepackage{iftex}
\ifPDFTeX
  \usepackage[T1]{fontenc}
  \usepackage[utf8]{inputenc}
  \usepackage{textcomp} % provide euro and other symbols
\else % if luatex or xetex
  \usepackage{unicode-math}
  \defaultfontfeatures{Scale=MatchLowercase}
  \defaultfontfeatures[\rmfamily]{Ligatures=TeX,Scale=1}
\fi

\titleformat{\section}
  {\normalfont\Large\scshape\bfseries}{\thesection}{1em}{}
  \titlespacing{\section}{0pt}{10pt}{0pt}
\titleformat{\subsection}
  {\normalfont\bfseries}{\thesection}{1em}{}
  \titlespacing{\subsection}{0pt}{6pt}{0pt}
\providecommand{\tightlist}{%
  \setlength{\itemsep}{0pt}\setlength{\parskip}{0pt}}\newenvironment{itemize*}%

  
\lhead{EC200 - Econometrics and Applications}

\setlength\parskip{0.10in}
\begin{document}
\thispagestyle{plain}
\singlespacing


EC200: Econometrics and Applications\hfill Fall 2021\\

%\vspace{1.5cm}
\begin{center}
\Large{\textbf{Lab 3: Linear Regression}}\\
\end{center}
\bigskip

\hypertarget{materials}{%
\section*{Materials}\label{materials}}
\addcontentsline{toc}{subsubsection}{Materials}

\begin{itemize}
\tightlist
\item
  \href{https://ec200f21.netlify.app/assignment/materials/graduation.dta}{\texttt{graduation.dta}}
\item
  Do-file template
  \href{https://ec200f21.netlify.app/assignment/materials/labtemplate_f21.do}{\texttt{labtemplate\_f21.do}}
\end{itemize}

\hypertarget{objectives}{%
\section*{Objectives}\label{objectives}}
\addcontentsline{toc}{subsubsection}{Objectives}

By the end of this tutorial you should be able to complete the following
tasks in Stata:

\begin{itemize}
\item
  Estimate and interpret a simple (two-variable) linear regression in
  levels, using continuous and binary variables, and use
  heteroskedasticity-robust standard errors.
\item
  Identify \(\hat{\beta_0}\), \(\hat{\beta_1}\), standard errors,
  \(SST\), \(SSE\), \(SSR\), and \(R^2\) in Stata output and interpret
  them
\item
  Calculate predicted values and residuals
\item
  Create scatter plots
\item
  Estimate a multivariate linear regression
\end{itemize}

\hypertarget{key-commands}{%
\section*{Key commands}\label{key-commands}}
\addcontentsline{toc}{subsubsection}{Key commands}

\begin{longtable}[]{@{}
  >{\raggedright\arraybackslash}p{(\columnwidth - 2\tabcolsep) * \real{0.56}}
  >{\raggedleft\arraybackslash}p{(\columnwidth - 2\tabcolsep) * \real{0.44}}@{}}
\toprule
command & description \\
\midrule
\endhead
Estimation commands & \\
\texttt{regress\ var1\ var2} & Estimate a regression, with \texttt{var1}
as the dependent variable and \texttt{var2} as the independent
variable(s) \\
\texttt{regress\ var1\ var2,\ robust} & Estimate a regression with
heteroskedasticity-robust standard errors \\
\texttt{correlate\ var1\ var2\ ...\ varn} & Calculate correlation
coefficients of all listed variables, from \texttt{var1} to
\texttt{varn}. \\
\texttt{graph\ twoway\ scatter\ var1\ var2} & make a scatter plot with
\texttt{var1} on the y-axis and \texttt{var2} on the x-axis. \\
Post-estimation commands\footnote{Post-estimation commands must be run
  \emph{immedately} after a regression, while the regression results are
  still held in your local variables} & \\
\texttt{predict\ newvar,\ xb} & Use estimated regression coefficients to
predict \(\widehat{y}\). It will generate \texttt{newvar}\footnote{Here,
  \texttt{newvar} equals
  \(\widehat{newvar_i} = \widehat{y_i} = \widehat{\beta_0} + \widehat{\beta_1}x_i\)} \\
\texttt{predict\ newvar,\ residuals} & Use estimated regression
coefficients to predict residuals, generating
\texttt{newvar}\footnote{Here, \texttt{newvar} equals
  \(\widehat{newvar_i} = u_i = y_i - \widehat{\beta_0} + \widehat{\beta_1}x_i\)} \\
Working with data, missing values & \\
\texttt{count\ if\ var1\ ==\ 1} & count observations if the expression
\texttt{var1\ ==\ 1} is true \\
\texttt{count\ if\ !missing(var1)} & count observations if \texttt{var1}
is not missing \\
\texttt{drop\ if\ missing(var1)} & drop all observations where
\texttt{var1} is missing \\
\texttt{tab\ var1,\ missing} & Include missing values in tabulation \\
\bottomrule
\end{longtable}



\hypertarget{lab-3-worksheet}{%
\section*{Lab 3 Exercise}\label{lab-3-worksheet}}
\addcontentsline{toc}{subsection}{Lab 3 Exercise}

\hypertarget{what-do-i-submit}{%
\subsection*{What do I submit?}\label{what-do-i-submit}}

\begin{itemize}
\tightlist
\item
  Your written up answers to exercise questions (1) - (13). This can be
  typed or written out then scanned (or photographed), in any reasonable
  format.
\item
  The do-file you've created that runs this analysis
\item
  A log file that contains the results from this exercise.
\end{itemize}

\hypertarget{questions}{%
\subsection*{Questions}\label{questions}}

%Today, we're going to look around at the graduation data set that we
%discussed in class,
%\href{../materials/graduation.dta}{\texttt{graduation.dta}}.

\begin{enumerate}
\def\labelenumi{\arabic{enumi}.}
\item
  Download the do-file template and data files. Personalize the file
  paths so that you can run it and open your \texttt{graduation.dta}
  file. You can also work with a blank data file if you're more
  comfortable - just make sure you remember to include commands to start
  and close your log file.
\item
  Take a look at \texttt{graduation.dta}. How many observations are
  there? What is the distribution of treatment arms?\footnote{There are
    a few variables here, including \texttt{treatment\_arm}}
\item
  There are six \emph{continuous} food security variables\footnote{Not
    \texttt{fsec7}, which is categorical, or \texttt{fsec} which is
    always equal to 1}. You can look for them with \texttt{lookfor\ fs}.
  Pick one variable and write out a population model to determine the
  relationship assignment to graduation and food security. For the rest
  of this lab, I refer to the variable you chose as
  \texttt{foodsecurity}. If that's going to irritate you, you can rename
  your variable like this: \texttt{rename\ fsec5\ foodsecurity}, using
  the variable name that you've chosen in place of \texttt{fsec5}.
\item
  Tabulate your food security value and check for missing observations.
  Drop any observations for which you have missing values of
  \texttt{foodsecurity} (see above for how to do this). How many
  observations are remaining?
\item
  Make a scatter plot of the relationship between your chosen food
  security variable and graduation (Include this in your submitted
  problem set). Is this easy to interpret? Calculate and report the
  associated correlation coefficient.
\item
  Conduct a t-test of whether the mean of \texttt{foodsecurity} is
  different between those who did and did not receive the graduation
  program\footnote{Hint; \texttt{ttest\ var1,by(var2)} will run a t-test
    of the mean of \texttt{var1} are equal for two groups determined by
    \texttt{var2}.}
\item
  Estimate the relationship between your chosen food security variable,
  \texttt{foodsecurity} and assignment to graduation,
  \texttt{graduation} using simple linear regression, with standard
  (homoskedasticity-assumed) standard errors. How do your t-statistics
  compare to what you found in the previous t-test? What was the impact
  of assignment to the graduation program on food security, based on
  your regression?
\item
  Re-estimate your regression, and this time adjust your standard errors
  to be heteroskedasticity-robust. Fill in the chart below with your
  estimates.
\end{enumerate}

\begin{longtable}[]{@{}lrrr@{}}
\toprule
Variable & Estimate & Variable & Estimate \\
\midrule
\endhead
\(\hat{\beta_0}\) & & \(\hat{\beta_1}\) & \\
\(R^2\) & & \(TSS\) & \\
\(ESS\) & & \(SSR\) & \\
d.f. & & \(SER\) & \\
\bottomrule
\end{longtable}

\begin{enumerate}
\def\labelenumi{\arabic{enumi}.}
\setcounter{enumi}{8}
\item
  After that regression estimate, generate a new variable,
  \texttt{predict\_fs} equal to the predicted value of your food
  security variable. Generate a second variable, \texttt{resid\_fs}
  equal to the residual.
\item
  What is the mean of each variable? How does the mean of
  \texttt{predict\_fs} compare to mean of \texttt{foodsecurity} in your
  sample?\footnote{If they differ, you should make sure you have dropped
    all missing values of \texttt{foodsecurity}! Try
    \texttt{sum\ predict\_fs\ foodsecurity} to see if the sample sizes
    are the same}
\item
  Examine the predicted value of your food security variable,
  \texttt{predict\_fs}, for the \emph{youngest} person in your
  sample.\footnote{Now is a good time to try out \texttt{lookfor\ age}}
  What is its residual?
\item
  When we estimate a linear regression with no coefficients, sometimes
  we'll say we are ``regressing on a constant.'' Regress
  \texttt{foodsecurity} \emph{only} on a constant. What is
  \(\hat{\beta_0}\), and how does it compare to overall mean?
\item
  For this final step, I'd like you to play around with the data. Pick
  \textbf{one} continuous dependent variable and \textbf{one} continuous
  \emph{or} binary independent variable.\footnote{Categorical variables
    that take on a just few observations, like the identity of your head
    of household, won't work here. You'll need to tabulate the variables
    to see what you're working with} You can look at the correlation
  between two variables, or you can look at the impact of one of the
  program dimensions (group coaching, group livelihood, etc) on an
  \emph{continuous} outcome of interest.

  \begin{enumerate}
  \def\labelenumii{\alph{enumii}.}
  \tightlist
  \item
    Write a population model you want to estimate.
  \item
    Estimate it using OLS, adjusting your standard errors to be
    heteroskedasticity robust. Write an equation that reflects your
    estimated model in the form
    \(\hat{y}=\hat{\beta_0} + \hat{\beta_1}x\), replacing \(y\) and
    \(x\) with your chosen varables and replacing \(\hat{\beta_0}\) and
    \(\hat{\beta_1}\) with your estimates.
  \item
    In 1-2 sentences, , what do your results tell you, collectively?
  \end{enumerate}
\end{enumerate}

\end{document}
