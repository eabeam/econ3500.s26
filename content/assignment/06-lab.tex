\documentclass[11pt]{article}




\usepackage[sfdefault]{FiraSans} %% option 'sfdefault' activates Fira Sans as the default text font
\usepackage[T1]{fontenc}
\renewcommand*\oldstylenums[1]{{\firaoldstyle #1}}

\usepackage{natbib}
\usepackage[french,english]{babel}
\usepackage{numprint}
\usepackage{multirow}
\usepackage{rotating}
\usepackage{fancyhdr}
\usepackage{booktabs}
\usepackage{multicol}
\usepackage{hyperref}\hypersetup{colorlinks=true}

\usepackage{amsmath,amssymb,amsfonts,textcomp}
\usepackage{color}
\usepackage{calc}
 \setlength{\tabcolsep}{8pt}
\usepackage{setspace}
\onehalfspacing
\usepackage{longtable}
\usepackage{graphicx}
\usepackage[margin=1in]{geometry}
\setlength{\parindent}{0pt}
\usepackage[bottom]{footmisc}
\pagestyle{fancy}
\usepackage{titlesec}
\usepackage{lipsum}
\usepackage{cancel}
\usepackage{multicol}

\usepackage{amsmath,amssymb}
\usepackage{lmodern}
\usepackage{iftex}
\ifPDFTeX
  \usepackage[T1]{fontenc}
  \usepackage[utf8]{inputenc}
  \usepackage{textcomp} % provide euro and other symbols
\else % if luatex or xetex
  \usepackage{unicode-math}
  \defaultfontfeatures{Scale=MatchLowercase}
  \defaultfontfeatures[\rmfamily]{Ligatures=TeX,Scale=1}
\fi

\titleformat{\section}
  {\normalfont\Large\scshape\bfseries}{\thesection}{1em}{}
  \titlespacing{\section}{0pt}{10pt}{0pt}
\titleformat{\subsection}
  {\normalfont\bfseries}{\thesection}{1em}{}
  \titlespacing{\subsection}{0pt}{6pt}{0pt}
\providecommand{\tightlist}{%
  \setlength{\itemsep}{0pt}\setlength{\parskip}{0pt}}\newenvironment{itemize*}%

  
\lhead{EC200 - Econometrics and Applications}
\rhead{Version: \today}
\setlength\parskip{0.10in}
\begin{document}
\thispagestyle{plain}
\singlespacing



EC200: Econometrics and Applications
%\vspace{1.5cm}
\begin{center}
\Large{\textbf{Lab 6: Internal Validity and LPM}}\\
\end{center}
\bigskip



\hypertarget{materials}{%
\subsection*{Materials}\label{materials}}
\addcontentsline{toc}{subsection}{Materials}

\begin{itemize}
\tightlist
\item
  \href{https://ec200f21.netlify.app/assignment/materials/cps_2016.dta}{\texttt{cps\_2016.dta}}
\item
  Do-file template
  \href{https://ec200f21.netlify.app/assignment/materials/labtemplate_f21.do}{\texttt{labtemplate.do}}
\end{itemize}

\hypertarget{objectives}{%
\section*{Objectives}\label{objectives}}
\addcontentsline{toc}{subsection}{Objectives}

Today we're going to keep working with
\href{https://ec200f21.netlify.app/assignment/materials/cps_2016.dta}{\texttt{cps\_2016.dta}}, which contains
information from the \href{https://cps.ipums.org/cps/}{2016 Current
Population Survey}.

By the end of this lab, you should be able to complete the following
tasks in Stata:

\begin{itemize}
\tightlist
\item
  Think about sample selection issues
\item
  Estimate and interpret linear probability models
\end{itemize}

\hypertarget{key-commands}{%
\section*{Key commands}\label{key-commands}}
\addcontentsline{toc}{subsection}{Key commands}

\begin{longtable}[]{@{}
  >{\raggedright\arraybackslash}p{(\columnwidth - 2\tabcolsep) * \real{0.56}}
  >{\raggedleft\arraybackslash}p{(\columnwidth - 2\tabcolsep) * \real{0.44}}@{}}
\toprule
command & description \\
\midrule
\endhead
\texttt{codebook\ var1} & Look at key details for \texttt{var1} \\
\texttt{clonevar\ var1\ =\ \ var2} & Make a new variable, \texttt{var1}
that duplicates \texttt{var2} (including labels!) \\
\texttt{\_pctile\ hourwages,per(99)} & Calcualte the 99th percentile of
hourly wages, and store as a local variable \\
\texttt{ret\ list} & Show locally stored variables (handy!) \\
\bottomrule
\end{longtable}

\hypertarget{lpm}{%
\section*{Linear Probability Models}\label{lpm}}
\addcontentsline{toc}{subsection}{Linear Probability Models}

What happens when our dependent variable is binary? We can use it
anyway! Using OLS with a binary dependent variable is called a
\textbf{linear probablity model}. There is lots of debate about whether
(and when) this is an okay idea, as it can lead to predcitions that are
below zero or greater than 1, and it violates homoskedasticity
assumptions. We can fix the latter by estimating
heteroskedasticity-robust standard errors, and the general consensus
\emph{seems} to be that usually, we're okay using a LPM. (Though we can
do better!)

What about interpretation? We intrepret coefficients are in
\textbf{percentage points} (not percents!)

Consider the following to see:

\(Married_i = \beta_0 + \beta_1 age_i + \beta_2 educ_i + u_i\)

\(\beta_1\) means that a 1-year increase in age is associated with a
\(\beta_1\) \textbf{percentage-point change} in the probability of being
married.

For great slides on this (and a deeper dive), check out
\href{https://nickch-k.github.io/EconometricsSlides/Week_08/Week_08_Limited_Dependent_Variables.html}{this
resource}!

\hypertarget{lab-6-worksheet}{%
\section*{Lab 6 Worksheet}\label{lab-6-worksheet}}

\hypertarget{what-do-i-submit}{%
\subsection*{What do I submit?}\label{what-do-i-submit}}

\begin{itemize}
\tightlist
\item
  Your written up answers to the exercise questions. This can be typed
  or written out then scanned (or photographed), in any reasonable
  format.
\item
  The do-file you've created that runs this analysis
\item
  A log file that contains the results from this exercise.
\end{itemize}

\hypertarget{exercises}{%
\subsection*{Exercises}\label{exercises}}

\begin{enumerate}
\def\labelenumi{\arabic{enumi}.}
\item
  Open Stata, start a new do-file (or bring in a template). Make sure
  you add code to start (and end) a log.
\item
  Open \texttt{cps\_2016.dta} and restrict the sample to adults (age
  18+) who are married (spouse present or absent). Drop anyone who
  reports ``NIU'' (not in universe) for labor force status. Confirm that
  you have \textbf{73,950} observations
\item
  Check work hours, weeks of work, and wage income for any weird recodes
  (that is, replace any 999999 values with missing values) generate the
  following variables, and ensure you have the correct means. You may
  want to use the \texttt{codebook} command to help
  (i.e.~\texttt{codebook\ uhrsworkly})
\end{enumerate}

\begin{verbatim}
    Variable |        Obs        Mean    Std. Dev.       Min        Max
-------------+---------------------------------------------------------
    wkswork1 |     73,950     34.0054     23.5977          0         52
  uhrsworkly |     51,921    40.19379    11.33071          1         99
-------------+---------------------------------------------------------
     incwage |     73,950    38947.58    64901.47          0    1259999

\end{verbatim}

\begin{enumerate}
\def\labelenumi{\arabic{enumi}.}
\setcounter{enumi}{3}
\item
  Generate a binary variable \texttt{female} equal to one if
  \texttt{sex\ ==\ 2}. Estimate the impact of \texttt{female} on wage
  income among married individuals. What is the interpretation on the
  coefficient?
\item
  If our objective is to measure the impact of gender on wage income
  among married individuals, is sample selection bias likely to be
  important? Why?  Is measurement error likely to be important? Why or why not?  If so, what is the likely impact of measurement error on your
  estimated coefficients?
\item
  Create a binary variable \texttt{lf} equal to 1 if an individual is in
  the labor force, and 0 otherwise. Estimate the impact of gender on
  labor force status. What is the interpretation of the coefficient?
  Estimate the impact of
\item
  What is the impact of being in the labor force on wage income? Based
  on this and the previous question, what is the implication for the
  direction of omitted variable bias when you estimated
  \(incwage = \beta_0 + \beta_1 female + u\) without controlling for
  it?
\item
  Re-estimate, including a control for \texttt{lf}:
  \(incwage = \beta_0 + \beta_1 female + \beta_2 lf + u\). Was your
  estimate correct?
\item
  Now, add your cleaned variable for usual hours worked to estimate
  \(incwage = \beta_0 + \beta_1 female + \beta_2 lf + \beta_3 uhrsworkly + u\).
  What is the interpretation of each coefficient?
\item
  Why does your regression not include all 73,850 people? What type of
  bias might this introduce?
\item
  Is measurement error likely to be important, and if so, for which
  variables? What is the likely impact of measurement error on your
  estimated coefficients?
\item
  Generate a new variable \texttt{uhrsNZ} that recodes all missing work
  hours values as zeros. You can expedite this with the
  \texttt{clonevar} command. Re-estimate the impact of gender, labor
  force status and \texttt{uhrsNZ} on wage income. What is the
  interpretation on \emph{each} coefficient? Why did it change?
\item
  Now, re-estimate but exclude \texttt{lf}:
  \(incwage = \beta_0 + \beta_1 female + \beta_3 uhrsworkly + u\). How
  do your results change? Conditional on including \texttt{female} and
  \texttt{uhrsworkly}, does it make sense to include \texttt{lf}?
\item
 Create a new variable that estimates log wages: $l\_incwage = log(incwage)$ Estimate the impact of gender on logged wage income, including a control for $uhrsworkly$. How does the sample size change, and why? What is the interpretation on each coefficient?
\item
  Calculate hourly wages, based on the cleaned variables. What is mean
  hourly wages for men and women?
\item
  Estimate the impact of gender on hourly wages for those with non-zero
  hourly wages, controlling for weekly work hours. Repeat then repeat to
  include all adults (Replace hourly wages with 0 for non-earners) How
  does the impact of gender on earnings compare?
\item
  Are there outlier values of $incwage$? Exclude observations that exceed the 99th  percentile in wages and re-estimate both equations. How does this
  affect your results?
\item
  Is measurement error likely to affect your dependent variable? Why or
  why not? If so, what are the implications?
\end{enumerate}

\end{document}
