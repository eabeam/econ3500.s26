\documentclass[11pt]{article}




\usepackage[sfdefault]{FiraSans} %% option 'sfdefault' activates Fira Sans as the default text font
\usepackage[T1]{fontenc}
\renewcommand*\oldstylenums[1]{{\firaoldstyle #1}}

\usepackage{natbib}
\usepackage[french,english]{babel}
\usepackage{numprint}
\usepackage{multirow}
\usepackage{rotating}
\usepackage{fancyhdr}
\usepackage{booktabs}
\usepackage{multicol}
\usepackage{hyperref}\hypersetup{colorlinks=true}

\usepackage{amsmath,amssymb,amsfonts,textcomp}
\usepackage{color}
\usepackage{calc}
 \setlength{\tabcolsep}{8pt}
\usepackage{setspace}
\onehalfspacing
\usepackage{longtable}
\usepackage{graphicx}
\usepackage[margin=1in]{geometry}
\setlength{\parindent}{0pt}
\usepackage[bottom]{footmisc}
\pagestyle{fancy}
\usepackage{titlesec}
\usepackage{lipsum}
\usepackage{cancel}
\usepackage{multicol}

\usepackage{amsmath,amssymb}
\usepackage{lmodern}
\usepackage{iftex}
\ifPDFTeX
  \usepackage[T1]{fontenc}
  \usepackage[utf8]{inputenc}
  \usepackage{textcomp} % provide euro and other symbols
\else % if luatex or xetex
  \usepackage{unicode-math}
  \defaultfontfeatures{Scale=MatchLowercase}
  \defaultfontfeatures[\rmfamily]{Ligatures=TeX,Scale=1}
\fi

\titleformat{\section}
  {\normalfont\Large\scshape\bfseries}{\thesection}{1em}{}
  \titlespacing{\section}{0pt}{10pt}{0pt}
\titleformat{\subsection}
  {\normalfont\bfseries}{\thesection}{1em}{}
  \titlespacing{\subsection}{0pt}{6pt}{0pt}
\providecommand{\tightlist}{%
  \setlength{\itemsep}{0pt}\setlength{\parskip}{0pt}}\newenvironment{itemize*}%

  
\lhead{EC200 - Econometrics and Applications}
\rhead{Version: \today}
\setlength\parskip{0.10in}
\begin{document}
\thispagestyle{plain}
\singlespacing


Version: \today \hfill Fall 2021\\
EC200: Econometrics and Applications
%\vspace{1.5cm}
\begin{center}
\Large{\textbf{Lab 2: Introduction to Do-Files}}\\
\end{center}
\bigskip

\hypertarget{materials}{%
\subsection*{Materials}\label{materials}}
\begin{itemize}
\tightlist
\item
  \href{../materials/acs2014_all.dta}{\texttt{acs2014\_all.dta}}
\item
  Do-file template
  \href{../materials/labtemplate_f20.do}{\texttt{labtemplate.do}}
\end{itemize}

\hypertarget{objectives}{%
\subsection*{Objectives}\label{objectives}}
\addcontentsline{toc}{subsubsection}{Objectives}

By the end of this tutorial you should be able to complete the following
tasks in Stata:

\begin{itemize}
\item
  Create and save a do-file
\item
  Explore variables and generate new ones
\item
  Be able to find help with Stata issues - find new commands, check and
  debug your work, etc.
\end{itemize}

\hypertarget{key-commands}{%
\subsection*{Key commands}\label{key-commands}}
\addcontentsline{toc}{subsubsection}{Key commands}

\begin{longtable}[]{@{}
  >{\raggedright\arraybackslash}p{(\columnwidth - 2\tabcolsep) * \real{0.56}}
  >{\raggedleft\arraybackslash}p{(\columnwidth - 2\tabcolsep) * \real{0.44}}@{}}
\toprule
command & description \\
\midrule
\endhead
Viewing data & \\
\texttt{tab\ var1} & tabulate one variable, \texttt{var1} \\
\texttt{tab\ var1,\ missing} & tabulate \texttt{var1}, include missing
values \\
\texttt{tab\ var1,\ nolabel} & tabulate \texttt{var1}, show values
rather than labels (if applicable) \\
& \\
Summarizing data & \\
\texttt{tabstat\ var1} & calculate mean of \texttt{var1} \\
\texttt{tabstat\ var1,by(var2)} & calculate mean of \texttt{var1}
separately for each value of \texttt{var2} \\
\texttt{tabstat\ var1,by(var2)\ stat(mean\ count\ p25\ p50\ p75)} &
calculate mean of \texttt{var1} separately for each value of
\texttt{var2}, with added statistics \\
Changing your data & \\
\texttt{gen\ newvar\ =var1} & generate a new variable, \texttt{newvar},
and set it equal to values of \texttt{var1} \\
\texttt{gen\ newvar\ =1\ if\ var2\ ==\ {[}exp{]}} & generate a new
variable, \texttt{newvar}, and set it equal to 1 if
\texttt{var2\ equals\ some\ expression,\ and\ missing\ otherwise} \\
\texttt{gen\ newvar\ =\ var2\ ==\ {[}exp{]}} & generate a new variable,
\texttt{newvar}, and set it equal to 1 if
\texttt{var2\ equals\ some\ expression,\ and\ 0\ otherwise} \\
\texttt{drop\ var1\ var2} & drop the variables \texttt{var1} and
\texttt{var2}. \\
\texttt{drop\ if\ {[}exp{]}} & drop observations for which \texttt{exp}
is true \\
\texttt{keep\ var1\ var2} & drop everything but \texttt{var1} and
\texttt{var2}. \\
\texttt{keep\ if\ {[}exp{]}} & keep observations \emph{only} if
\texttt{exp} is true \\
Displaying your data & \\
\texttt{graph\ twoway\ histogram\ var1} & make a histogram for
\texttt{var1.} Check help files for more options \\
\bottomrule
\end{longtable}

Looking for more examples? Check out these
\href{https://geocenter.github.io/StataTraining/portfolio/01_resource/}{\textbf{Stata
Cheat Sheets}}

Suppose I asked you to recreate your analysis from our last lab. How
long would it take you? If you used a do-file, you would just have to
click a button, because your analysis would be replicable. We're going
to learn about the glory of do-files and a few other descriptive
statistics tricks.

The instant gratification of the Command window is tempting, but getting
comfortable with do-files will save you lots of time, make collaboration
easier, and reduce errors!

\hypertarget{aside-bad-documentation-big-problems}{%
\subsubsection*{Aside: Bad documentation, big
problems}\label{aside-bad-documentation-big-problems}}

\begin{quote}
For an economist, the five most terrifying words in the English language
are: I can't replicate your results.But for economists Carmen Reinhart
and Ken Rogoff of Harvard, there are seven even more terrifying ones: I
think you made an Excel error.

--
\href{https://www.theatlantic.com/business/archive/2013/04/forget-excel-this-was-reinhart-and-rogoffs-biggest-mistake/275088/}{Matthew
O'Brien, The Atlantic (18 April 2013)}
\end{quote}

A summary from
\href{https://theconversation.com/the-reinhart-rogoff-error-or-how-not-to-excel-at-economics-13646}{The
Conversation, (22 April, 2013)}

\begin{quote}
Reinhart and Rogoff's work showed average real economic growth slows (a
0.1\% decline) when a country's debt rises to more than 90\% of gross
domestic product (GDP) -- and this 90\% figure was employed repeatedly
in political arguments over high-profile austerity measures\ldots{}

The most serious was that, in their Excel spreadsheet, Reinhart and
Rogoff had not selected the entire row when averaging growth figures:
they omitted data from Australia, Austria, Belgium, Canada and Denmark.

In other words, they had accidentally only included 15 of the 20
countries under analysis in their key calculation.

When that error was corrected, the ``0.1\% decline'' data became a 2.2\%
average increase in economic growth.

So the key conclusion of a seminal paper, which has been widely quoted
in political debates in North America, Europe Australia and elsewhere,
was invalid.
\end{quote}


\hypertarget{do-files-and-the-do-file-editor}{%
\subsubsection*{Do-files and the do-file
editor}\label{do-files-and-the-do-file-editor}}

You can get pretty far in Stata relying on the Command and Review
window, but we may want a record of the commands we want to run for our
analysis. One thing that makes Stata different from a program like Excel
is that you can create do-files, essentially small programs that will
run your analysis again and again, in exactly the same way. For
econometric analysis this is CRUCIAL.

A do-file can be written in any text file and then saved with the
extension \texttt{.do}, but we'll use the do-file editor. You can start
a new do-file by clicking on the do-file button. Or, you can open the
do-file template.

The do-file editor is where we will write our programs, and it has some
nice color coding to help us avoid mistakes. For your problem sets and
papers, you must ALWAYS submit a do-file along with your results. Some
people will like to practice in the Command window and then copy the
commands they're satisfied with to the do-file, while others will prefer
to work entirely in the do-file. It's your call, though the second one
is a little less risky.

\hypertarget{comment-comment-comment}{%
\paragraph{Comment, comment, comment}\label{comment-comment-comment}}

Do-files are used to record your past work and possibly to share your
work with others. It's important to properly \textbf{document} your work
using comments. There are three ways to comment

\begin{enumerate}
\def\labelenumi{\arabic{enumi}.}
\item
  Comment the whole line with an asterisk
\item
  Comment the whole line or part of a line with two forward slashes
  (\texttt{//})
\item
  Use slash-asterisk to open (\texttt{/*}) and close (\texttt{*/}) a
  comment section
\end{enumerate}

The do-file editor will turn all your comments green so you don't get
confused.

\hypertarget{programming-tips}{%
\subsubsection*{Programming tips}\label{programming-tips}}

\begin{itemize}
\item
  \textbf{Put everything in a do-file!} An important feature of any good
  research project is that the results should be reproducible. For Stata
  the easiest way to do this is to create a text file that lists all
  your commands in order, so anyone can re-run all your Stata work on a
  project anytime. Such text files that are produced within Stata or
  linked to Stata are called do-files, because they have an extension
  .do (like \texttt{intro\_exercise.do}). These files feed commands
  directly into Stata without you having to type or copy them into the
  command window.

  Imagine you're just about done with the analysis for your research
  paper. While working on the final regression, you discover that one of
  your variables wasn't cleaned properly, and you need to drop some
  outliers from the data. Do you correct it and redo everything from
  scratch? Could you even do that? How long would it take?

  With a set of do-files, all you have to do is correct the variable
  early in the code, and re-run everything. If your code is quick, it
  will take just a few minutes. Easy!

  An added bonus is that having do-files makes it very easy to fix your
  typos, re-order commands, and create more complicated chains of
  commands that wouldn't work otherwise. You can now quickly reproduce
  your work, correct it, adjust it, and build on it.
\item
  \textbf{Log your results.} Maintaining logs can help you quickly
  retrieve results and serve as a record of past work in case you
  accidentally overwrite commands. Logs contain the commands \emph{and}
  the results.
\item
  \textbf{Never overwrite your original files.} A good do-file structure
  starts with your original, raw data, then cleans and analyzes it to
  get your final results. A ``master'' do-file can piece all these
  together.
\item
  \textbf{Replicability is key.} Your code should be replicable to
  someone else who picks up your raw files and code.
\item
  \textbf{Comment, comment, comment!} Clear commenting is essential to
  help others understand your code and to remember what you did.
\end{itemize}

\hypertarget{finding-new-commands}{%
\subsubsection*{Finding new commands}\label{finding-new-commands}}

One of the strengths of Stata is that complicated processes can be
completed with simple commands. One of its weaknesses is that it's not
always obvious what those specific commands are. In our problem sets and
your research paper, you will (I promise) have to calculate or estimate
something in a way we haven't covered.

\begin{itemize}
\item
  Stata help file: \texttt{help\ command}
\item
  Search Stata documentation: \texttt{finidit\ keyword}
\item
  Google the thing you are trying to do: \emph{i.e.: make scatterplot
  Stata, turn rows into columns Stata, cluster standard errors Stata,
  etc.}
\end{itemize}
\clearpage
\hypertarget{lab-2}{%
\section*{Lab Exercise 2}\label{lab-2}}
\addcontentsline{toc}{subsection}{Lab Exercise 2}

\hypertarget{what-do-i-submit}{%
\subsubsection*{What do I submit?}\label{what-do-i-submit}}

\begin{itemize}
\tightlist
\item
  Your written up answers to exercise questions (1) - (12). This can be
  typed or written out then scanned (or photographed), in any reasonable
  format.
\item
  The do-file you've created that runs this analysis and log file with results

\end{itemize}
\vspace{-0.35in}
\hypertarget{questions}{%
\subsubsection*{Questions}\label{questions}}
\vspace{-0.15in}

\begin{enumerate}
\def\labelenumi{\arabic{enumi}.}
\item
  Download \href{materials/acs2014_all.dta}{\texttt{acs2014\_all.dta}}
  and \href{materials/labtemplate_f21.do}{\texttt{labtemplate.do}} by
  clicking the links.
\item
  Move these files to wherever you store your class materials.
\item
  Open \texttt{labtemplate\_f21.do} and run it. Does it work? Probably
  not! Fix it.
\item
  Drop some variables we don't need right now: \texttt{gq},
  \texttt{serial}, and \texttt{hhwt}. How many variables remain?
\item
  What is the age distribution of the sample? Specifically, report the
  mean, median, minimum, and maximum age of the sample.
\item
  Because very young workers might be still in school, drop anyone in
  your sample who is less than 23 years old. (Maintain this sample
  restriction for the rest of the lab). How many people are left in your
  sample?
\item
  Generate a new variable, \texttt{lt35} that is equal to one if a
  person is less than 35 years old and 0 otherwise. What is the mean of
  \texttt{lt35,} and what is its interpretation?
\item
  Using the \texttt{tabstat} command, find the average income and wages
  for those under age 35 and those at least age 35. How does it compare
  to median income and wages for each group?
\item
  Using the \texttt{tabstat} command, find the average income and wages
  for men and women.
\item
  There are several reasons why men might earn more than women. Suppose
  you hypothesized that that men have completed more education than
  women; and workers with higher education levels earn more. We will
  test this in two ways.

  \begin{enumerate}
  \def\labelenumii{\arabic{enumii}.}
  \item
    First, generate a variable equal to one if a person has completed at
    least some post-secondary education, and zero otherwise. What is the
    mean of this variable?
  \item
    What share of men have at least some post-secondary education? What
    about women?
  \item
    We can also see if gender-wage gaps are bigger for lower vs.
    higher-educated workers. For those without post-secondary education,
    what is the average wage gap? For those with post-secondary
    education, what is the average wage gap?
  \end{enumerate}
\item
  Name \textbf{two} additional reasons that may explain why men's income
  is higher than women's income on average. How would you test each one?
  \emph{You do not have to actually do this test, just describe in as
  much detail as possible. You can assume you have additional data
  beyond what is provided here.}
\item
  Make two histograms, one of the income distribution for men and one of
  the income distribution for women. Make sure the y-axis indicates the
  ``fraction'' of individuals, not the density. Copy and paste it into
  your responses.
\end{enumerate}

\end{document}
